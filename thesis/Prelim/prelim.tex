% This file contains all the necessary setup and commands to create
% the preliminary pages according to the buthesis.sty option.

\title{PULER}

\author{Bassel EL Mabsout}

% Type of document prepared for this degree:
%   1 = Master of Science thesis,
%   2 = Doctor of Philosophy dissertation.
\degree=1

\prevdegrees{B.S., American University of Beirut, 2015}

\department{Department of Computer Science}

% Degree year is the year the diploma is expected, and defense year is
% the year the dissertation is written up and defended. Often, these
% will be the same, except for January graduation, when your defense
% will be in the fall of year X, and your graduation will be in
% January of year X+1
\defenseyear{2023}
\degreeyear{2023}

% For each reader, specify appropriate label {First, Second, Third},
% then name, and title. IMPORTANT: The title should be:
%   "Professor of Electrical and Computer Engineering",
% or similar, but it MUST NOT be:
%   Professor, Department of Electrical and Computer Engineering"
% or you will be asked to reprint and get new signatures.
% Warning: If you have more than five readers you are out of luck,
% because it will overflow to a new page. You may try to put part of
% the title in with the name.
\reader{First}{Renato Mancuso, PhD}{Assistant Professor of Computer Science}
\reader{Second}{Marco Gaboardi, PhD}{Associate Professor of Computer Science}

% The Major Professor is the same as the first reader, but must be
% specified again for the abstract page. Up to 4 Major Professors
% (advisors) can be defined. 
\numadvisors=1
\majorprof{Renato Mancuso, PhD}{{Professor of Computer Science}}
%\majorprofc{First M. Last, PhD}{{Professor of Astronomy}}
%\majorprofd{First M. Last, PhD}{{Professor of Biomedical Engineering}}

%%%%%%%%%%%%%%%%%%%%%%%%%%%%%%%%%%%%%%%%%%%%%%%%%%%%%%%%%%%%%%%%  

%                       PRELIMINARY PAGES
% According to the BU guide the preliminary pages consist of:
% title, copyright (optional), approval,  acknowledgments (opt.),
% abstract, preface (opt.), Table of contents, List of tables (if
% any), List of illustrations (if any). The \tableofcontents,
% \listoffigures, and \listoftables commands can be used in the
% appropriate places. For other things like preface, do it manually
% with something like \newpage\section*{Preface}.

% This is an additional page to print a boxed-in title, author name and
% degree statement so that they are visible through the opening in BU
% covers used for reports. This makes a nicely bound copy. Uncomment only
% if you are printing a hardcopy for such covers. Leave commented out
% when producing PDF for library submission.
%\buecethesistitleboxpage

% Make the titlepage based on the above information.  If you need
% something special and can't use the standard form, you can specify
% the exact text of the titlepage yourself.  Put it in a titlepage
% environment and leave blank lines where you want vertical space.
% The spaces will be adjusted to fill the entire page.
\maketitle
\cleardoublepage

% The copyright page is blank except for the notice at the bottom. You
% must provide your name in capitals.
\copyrightpage
\cleardoublepage

% Now include the approval page based on the readers information
% Once the approval page is approved by the Mugar Library staff, please
% comment out the "\approvalpagewithcomment" line and uncomment "\approvalpage"
% \approvalpagewithcomment
\approvalpage
\cleardoublepage

% Here goes your favorite quote. This page is optional.
\newpage
%\thispagestyle{empty}
\phantom{.}
\vspace{4in}

\begin{singlespace}
\begin{quote}
  \textit{Puler:}\\
  \textit{One who pules; one who whines or complains}\\*
  \hfill{Webster Dictionary}
\end{quote}
\end{singlespace}

% \vspace{0.7in}
%
% \noindent
% [The descent to Avernus is easy; the gate of Pluto stands open night
% and day; but to retrace one's steps and return to the upper air, that
% is the toil, that the difficulty.]

\cleardoublepage

% The acknowledgment page should go here. Use something like
% \newpage\section*{Acknowledgments} followed by your text.
\newpage


% The abstractpage environment sets up everything on the page except
% the text itself.  The title and other header material are put at the
% top of the page, and the supervisors are listed at the bottom.  A
% new page is begun both before and after.  Of course, an abstract may
% be more than one page itself.  If you need more control over the
% format of the page, you can use the abstract environment, which puts
% the word "Abstract" at the beginning and single spaces its text.

\begin{abstractpage}
% ABSTRACT
We address the prevalent challenge of Algebraic Data Type duplication in compiler implementations, which results in increased effort, diminished functionality, and complications in synchronizing language constructs across the compiler. To investigate a novel design solution, we present Tree Shaping, a solution to the expression problem. We then implement an experimental compiler using Tree Shaping and examine its potential implications. This compiler processes programs written in $\PULER$, an ML-based programming language that boasts distinct features such as unification rules for type mismatches. Contrary to traditional compilers that terminate and generate an error when encountering a type mismatch, $\PULER$ regards type mismatches as first-class citizens.
\end{abstractpage}
\cleardoublepage

% Now you can include a preface. Again, use something like
% \newpage\section*{Preface} followed by your text

% Table of contents comes after preface
\tableofcontents
\cleardoublepage

% If you do not have tables, comment out the following lines
% \newpage
% \listoftables
% \cleardoublepage

% If you have figures, uncomment the following line
% \newpage
% \listoffigures
% \cleardoublepage

% List of Abbrevs is NOT optional (Martha Wellman likes all abbrevs listed)
\chapter*{List of Abbreviations}

\begin{center}
  \begin{tabular}{lll}
    \hspace*{2em} & \hspace*{1in} & \hspace*{4.5in} \\
    AST  & \dotfill & Abstract Syntax Tree \\
    ADT   & \dotfill & Algebraic Data Type \\
    GHC  & \dotfill & Glasgow Haskell Compiler \\
    BNF & \dotfill & Backus-Naur Form \\
    DSL  & \dotfill & Domain Specific Languages \\
    REPL  & \dotfill & Read Eval Print Loop \\
    ML  & \dotfill & Meta-Language \\
  \end{tabular}
\end{center}
\cleardoublepage

% END OF THE PRELIMINARY PAGES

\newpage
\endofprelim
