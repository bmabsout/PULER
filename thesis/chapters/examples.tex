\chapter{Examples}
\label{examples}
\thispagestyle{myheadings}


\section{Example Program}
\begin{minted}{haskell}
    fib = fix fib \x ->
        if x == 0
        then 0
        else if x == 1
             then 1
             else (fib (x-1)) + (fib (x-2));
    
    main = (fib 8);
    >>
    21
\end{minted}

\section{Example Commands}
    \subsection{REPL example}
    Open a repl and run expressions as well as check their types:
    \begin{minted}{bash}
        > PULER repl
        Welcome to PULER!
    \end{minted}
    \begin{minted}{haskell}
        λ> x = 1
        x = 1;
        λ> y = \z -> z
        y = \z -> <body>;
        λ> :t (y x)
        (y: Int -> Int x: Int): Int
        λ> (y x) + 1
        2
        λ> (Print "test")
        "test"
        {}
    \end{minted}

    
    \subsection{Compilation stages example}\label{compiler_stages}
    Compile an example *.pul file implementing the factorial function
    \mint{bash}|    > PULER example.pul|
    \begin{minted}[fontsize=\footnotesize]{haskell}
        -----INPUT-----
        fact = fix f \x ->
                      if x == 0
                      then 1
                      else let nextFac = x * (f (x - 1));
                               p = (Print (Int2Str nextFac))
                           in nextFac;
        main = let g = (fact 8) in {};
        
        -----PARSED-----
        fact = fix f \x -> if ((EqInt x) 0)
                           then 1
                           else let nextFac = ((Mul x) (f ((Sub x) 1)));
                                    p = (Print (Int2Str nextFac));
                                in nextFac;
        main = let g = (fact 8);
               in {};
        -----RENAMED-----
        fact = fix f \x -> if ((EqInt x) 0)
                           then 1
                           else let nextFac = ((Mul x) (f ((Sub x) 1)));
                                    p = (Print (Int2Str nextFac));
                                in nextFac;
        main = let g = (fact 8);
               in {};
        -----INFERRED-----
        fact = fix f \x -> if ((EqInt x:Int):Int->Bool 0:Int):Bool
                           then 1:Int
                           else let nextFac = ((Mul x:Int):Int->Int
                                                 (f:Int->Int ((Sub x:Int):Int->Int
                                                                1:Int):Int):Int):Int;
                                    p = (Print (Int2Str nextFac:Int):String):{};
                                in nextFac:Int:Int:Int:Int->Int;
        main = let g = (fact:Int->Int 8:Int):Int;
               in {}:{}:{};:{}
        -----CHECKED-----
        Typechecks!
        -----EVALUATED-----
        1
        2
        6
        24
        120
        720
        5040
        40320
        {}
        -----EMITTED-----
        fact = fix(lambda f: lambda x:(
            (
                1
            ) if (
                EqInt(x)(0)
            ) else (
                let(
                  nextFac = Mul(x)(f(Sub(x)(1))),
                inn = lambda nextFac:(
                  let(
                    p = Print(Int2Str(nextFac)),
                  inn = lambda p:(
                    nextFac
                  ))
                ))
            )
        ))
        main = let(
          g = fact(8),
        inn = lambda g:(
          unit
        ))
    \end{minted}